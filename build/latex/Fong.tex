%% Generated by Sphinx.
\def\sphinxdocclass{report}
\documentclass[letterpaper,10pt,english]{sphinxmanual}
\ifdefined\pdfpxdimen
   \let\sphinxpxdimen\pdfpxdimen\else\newdimen\sphinxpxdimen
\fi \sphinxpxdimen=.75bp\relax

\PassOptionsToPackage{warn}{textcomp}
\usepackage[utf8]{inputenc}
\ifdefined\DeclareUnicodeCharacter
 \ifdefined\DeclareUnicodeCharacterAsOptional
  \DeclareUnicodeCharacter{"00A0}{\nobreakspace}
  \DeclareUnicodeCharacter{"2500}{\sphinxunichar{2500}}
  \DeclareUnicodeCharacter{"2502}{\sphinxunichar{2502}}
  \DeclareUnicodeCharacter{"2514}{\sphinxunichar{2514}}
  \DeclareUnicodeCharacter{"251C}{\sphinxunichar{251C}}
  \DeclareUnicodeCharacter{"2572}{\textbackslash}
 \else
  \DeclareUnicodeCharacter{00A0}{\nobreakspace}
  \DeclareUnicodeCharacter{2500}{\sphinxunichar{2500}}
  \DeclareUnicodeCharacter{2502}{\sphinxunichar{2502}}
  \DeclareUnicodeCharacter{2514}{\sphinxunichar{2514}}
  \DeclareUnicodeCharacter{251C}{\sphinxunichar{251C}}
  \DeclareUnicodeCharacter{2572}{\textbackslash}
 \fi
\fi
\usepackage{cmap}
\usepackage[T1]{fontenc}
\usepackage{amsmath,amssymb,amstext}
\usepackage{babel}
\usepackage{times}
\usepackage[Bjarne]{fncychap}
\usepackage{sphinx}

\usepackage{geometry}

% Include hyperref last.
\usepackage{hyperref}
% Fix anchor placement for figures with captions.
\usepackage{hypcap}% it must be loaded after hyperref.
% Set up styles of URL: it should be placed after hyperref.
\urlstyle{same}
\addto\captionsenglish{\renewcommand{\contentsname}{目录}}

\addto\captionsenglish{\renewcommand{\figurename}{Fig.}}
\addto\captionsenglish{\renewcommand{\tablename}{Table}}
\addto\captionsenglish{\renewcommand{\literalblockname}{Listing}}

\addto\captionsenglish{\renewcommand{\literalblockcontinuedname}{continued from previous page}}
\addto\captionsenglish{\renewcommand{\literalblockcontinuesname}{continues on next page}}

\addto\extrasenglish{\def\pageautorefname{page}}

\setcounter{tocdepth}{0}


	    % Use some font with UTF-8 support with XeLaTeX
        \usepackage[UTF8]{ctex}
     

\title{Fong Documentation}
\date{May 21, 2018}
\release{alpha}
\author{fong}
\newcommand{\sphinxlogo}{\vbox{}}
\renewcommand{\releasename}{Release}
\makeindex

\begin{document}

\maketitle
\sphinxtableofcontents
\phantomsection\label{\detokenize{index::doc}}


\begin{sphinxadmonition}{note}{Note:}
\sphinxstylestrong{文中可能存在错误,欢迎 PR。}
\end{sphinxadmonition}


\chapter{C/C++}
\label{\detokenize{cpp/index:c-c}}\label{\detokenize{cpp/index::doc}}\label{\detokenize{cpp/index:id1}}

\section{main函数}
\label{\detokenize{cpp/01_main:main}}\label{\detokenize{cpp/01_main::doc}}

\subsection{返回值}
\label{\detokenize{cpp/01_main:id1}}
C++ main函数的返回值必须是 \sphinxcode{\sphinxupquote{int}} ,即整型类型。在大多数系统中,main 的返回值被用来指示状态,返回值 \sphinxcode{\sphinxupquote{0}} 表示执行成功,非0的返回值含义由系统定义,通常用来指出错误类型。

Windows系统下运行可执行文件(如launch.exe)可以直接忽略其扩展名 \sphinxcode{\sphinxupquote{.exe}} :

\fvset{hllines={, ,}}%
\begin{sphinxVerbatim}[commandchars=\\\{\}]
\PYG{n}{launch}
\end{sphinxVerbatim}

Unix系统下需要使用全文件名,包括扩展名:

\fvset{hllines={, ,}}%
\begin{sphinxVerbatim}[commandchars=\\\{\}]
./a.out
\end{sphinxVerbatim}

访问main函数返回之后的方法依赖于系统。在Windows和Unix系统中,执行完一个程序之后,都可以通过 \sphinxcode{\sphinxupquote{echo}} 命令来获取返回值。

Windows:

\fvset{hllines={, ,}}%
\begin{sphinxVerbatim}[commandchars=\\\{\}]
\PYG{n+nb}{echo} \PYGZpc{}ERRORLEVEL\PYGZpc{}
\end{sphinxVerbatim}

Unix:

\fvset{hllines={, ,}}%
\begin{sphinxVerbatim}[commandchars=\\\{\}]
\PYG{n+nb}{echo} \PYG{n+nv}{\PYGZdl{}?}
\end{sphinxVerbatim}


\subsection{处理命令行选项}
\label{\detokenize{cpp/01_main:id2}}
main函数的形参列表有两种形式:

\fvset{hllines={, ,}}%
\begin{sphinxVerbatim}[commandchars=\\\{\}]
\PYG{k+kt}{int} \PYG{n+nf}{main}\PYG{p}{(}\PYG{k+kt}{int} \PYG{n}{argc}\PYG{p}{,} \PYG{k+kt}{char} \PYG{o}{*}\PYG{n}{argv}\PYG{p}{[}\PYG{p}{]}\PYG{p}{)}\PYG{p}{\PYGZob{}} \PYG{p}{.}\PYG{p}{.}\PYG{p}{.} \PYG{p}{\PYGZcb{}}

\PYG{k+kt}{int} \PYG{n+nf}{main}\PYG{p}{(}\PYG{k+kt}{int} \PYG{n}{argc}\PYG{p}{,} \PYG{k+kt}{char} \PYG{o}{*}\PYG{o}{*}\PYG{n}{argv}\PYG{p}{)}\PYG{p}{\PYGZob{}} \PYG{p}{.}\PYG{p}{.}\PYG{p}{.} \PYG{p}{\PYGZcb{}}
\end{sphinxVerbatim}

第一种形参 \sphinxcode{\sphinxupquote{*argv{[}{]}}} 中,\sphinxcode{\sphinxupquote{argv}} 是一个数组,它的元素是指向C风格的字符串的指针;
第二种形参 \sphinxcode{\sphinxupquote{**argv}} 中,\sphinxcode{\sphinxupquote{argv}} 指向 \sphinxcode{\sphinxupquote{char*}} 。
参数 \sphinxcode{\sphinxupquote{argc}} 表示数组中字符串的数量。

当实参传给main函数之后,\sphinxcode{\sphinxupquote{argv}} 的第一个元素指向程序的名字或者一个空字符串,接下来的元素依次传递命令行提供的实参。最后一个指针之后的元素值保证为0。
例如,执行:

\fvset{hllines={, ,}}%
\begin{sphinxVerbatim}[commandchars=\\\{\}]
\PYG{n}{launch} \PYG{o}{\PYGZhy{}}\PYG{n}{d} \PYG{o}{\PYGZhy{}}\PYG{n}{o} \PYG{n}{ofile} \PYG{n}{data}
\end{sphinxVerbatim}

\sphinxcode{\sphinxupquote{launch}} 是可执行文件。那么, \sphinxcode{\sphinxupquote{argc=5}} ,\sphinxcode{\sphinxupquote{argv}} 包含如下的C风格字符串:

\fvset{hllines={, ,}}%
\begin{sphinxVerbatim}[commandchars=\\\{\},numbers=left,firstnumber=1,stepnumber=1]
\PYG{n}{argv}\PYG{p}{[}\PYG{l+m+mi}{0}\PYG{p}{]} \PYG{o}{=} \PYG{l+s}{\PYGZdq{}}\PYG{l+s}{launch}\PYG{l+s}{\PYGZdq{}}\PYG{p}{;}
\PYG{n}{argv}\PYG{p}{[}\PYG{l+m+mi}{1}\PYG{p}{]} \PYG{o}{=} \PYG{l+s}{\PYGZdq{}}\PYG{l+s}{\PYGZhy{}d}\PYG{l+s}{\PYGZdq{}}\PYG{p}{;}
\PYG{n}{argv}\PYG{p}{[}\PYG{l+m+mi}{2}\PYG{p}{]} \PYG{o}{=} \PYG{l+s}{\PYGZdq{}}\PYG{l+s}{\PYGZhy{}o}\PYG{l+s}{\PYGZdq{}}\PYG{p}{;}
\PYG{n}{argv}\PYG{p}{[}\PYG{l+m+mi}{3}\PYG{p}{]} \PYG{o}{=} \PYG{l+s}{\PYGZdq{}}\PYG{l+s}{ofile}\PYG{l+s}{\PYGZdq{}}\PYG{p}{;}
\PYG{n}{argv}\PYG{p}{[}\PYG{l+m+mi}{4}\PYG{p}{]} \PYG{o}{=} \PYG{l+s}{\PYGZdq{}}\PYG{l+s}{data}\PYG{l+s}{\PYGZdq{}}\PYG{p}{;}
\PYG{n}{argv}\PYG{p}{[}\PYG{l+m+mi}{5}\PYG{p}{]} \PYG{o}{=} \PYG{l+s}{\PYGZdq{}}\PYG{l+s}{0}\PYG{l+s}{\PYGZdq{}}\PYG{p}{;}
\end{sphinxVerbatim}

\begin{sphinxadmonition}{note}{Note:}
当使用 \sphinxcode{\sphinxupquote{argv}} 中的实参时,实参是从 \sphinxcode{\sphinxupquote{argv{[}1{]}}} 开始的; \sphinxcode{\sphinxupquote{argv{[}0{]}}} 保存的是程序名,而非用户输入。
\end{sphinxadmonition}


\subsection{参考资料}
\label{\detokenize{cpp/01_main:id3}}
《C++ Primer 第5版 中文版》 Page 2, Page 197。


\chapter{Python}
\label{\detokenize{python/index:python}}\label{\detokenize{python/index::doc}}

\chapter{Linux/Shell}
\label{\detokenize{linux/index:linux-shell}}\label{\detokenize{linux/index::doc}}

\chapter{机器学习}
\label{\detokenize{machineLearning/index::doc}}\label{\detokenize{machineLearning/index:id1}}

\chapter{深度学习}
\label{\detokenize{deepLearning/index::doc}}\label{\detokenize{deepLearning/index:id1}}

\chapter{资源链接}
\label{\detokenize{link/index::doc}}\label{\detokenize{link/index:id1}}

\section{Github Page}
\label{\detokenize{link/index:github-page}}
\sphinxurl{https://fongyk.github.io/}


\section{arXiv}
\label{\detokenize{link/index:arxiv}}
\sphinxurl{https://arxiv.org/}


\section{Read the Docs}
\label{\detokenize{link/index:read-the-docs}}
\sphinxurl{https://readthedocs.org/}


\section{C++ Reference}
\label{\detokenize{link/index:c-reference}}
\sphinxurl{http://www.cplusplus.com/reference/}


\section{Numpy}
\label{\detokenize{link/index:numpy}}
\sphinxurl{http://cs231n.github.io/python-numpy-tutorial/}


\section{Pytorch}
\label{\detokenize{link/index:pytorch}}
Tutorials: \sphinxurl{https://pytorch.org/tutorials/}

Docs: \sphinxurl{https://pytorch.org/docs/master/index.html}


\section{Standford University Lectures}
\label{\detokenize{link/index:standford-university-lectures}}
CS229: \sphinxurl{http://cs229.stanford.edu/syllabus.html}

CS231: \sphinxurl{http://cs231n.github.io/}


\section{ShareLatex}
\label{\detokenize{link/index:sharelatex}}
\sphinxurl{https://www.sharelatex.com/login}


\section{PlanetB}
\label{\detokenize{link/index:planetb}}
\sphinxurl{http://www.planetb.ca/syntax-highlight-word}


\section{Vision Open Source Library}
\label{\detokenize{link/index:vision-open-source-library}}
检索: \sphinxurl{http://yael.gforge.inria.fr/index.html}

特征: \sphinxurl{http://www.vlfeat.org/index.html}


\section{牛客网}
\label{\detokenize{link/index:id2}}
\sphinxurl{https://www.nowcoder.com/}


\chapter{实用软件}
\label{\detokenize{softwares/index::doc}}\label{\detokenize{softwares/index:id1}}

\section{Listary}
\label{\detokenize{softwares/index:listary}}
\begin{sphinxadmonition}{note}{Note:}
Windows下快速查找文件及应用程序
\end{sphinxadmonition}

\sphinxurl{http://www.listary.com/}


\section{FreeCommander}
\label{\detokenize{softwares/index:freecommander}}
\begin{sphinxadmonition}{note}{Note:}
Windows下的资源管理器
\end{sphinxadmonition}

\sphinxurl{https://freecommander.com/en/summary/}


\section{MobaXterm}
\label{\detokenize{softwares/index:mobaxterm}}
\begin{sphinxadmonition}{note}{Note:}
Windows下连接服务器的终端
\end{sphinxadmonition}

\sphinxurl{https://mobaxterm.mobatek.net/}


\section{TeamViewer}
\label{\detokenize{softwares/index:teamviewer}}
\begin{sphinxadmonition}{note}{Note:}
远程连接
\end{sphinxadmonition}

\sphinxurl{https://www.teamviewer.com/zhCN/}


\section{Notepad++}
\label{\detokenize{softwares/index:notepad}}
\begin{sphinxadmonition}{note}{Note:}
强大的文本阅读/编辑器
\end{sphinxadmonition}

\sphinxurl{https://notepad-plus-plus.org/}


\chapter{其他}
\label{\detokenize{else/index::doc}}\label{\detokenize{else/index:id1}}

\section{rst语法测试}
\label{\detokenize{else/01_test_code:rst}}\label{\detokenize{else/01_test_code::doc}}
\sphinxcode{\sphinxupquote{makefile}} 规则:

\fvset{hllines={, ,}}%
\begin{sphinxVerbatim}[commandchars=\\\{\}]
\PYG{n+nf}{target ... }\PYG{o}{:} \PYG{n}{prerequisites} ...
    \PYG{n+nb}{command}
    ...
    ...
\end{sphinxVerbatim}

下面是几个定义:
\begin{description}
\item[{target}] \leavevmode
可以是一个object file(目标文件),也可以是一个执行文件,还可以是一个标签(label)。对
于标签这种特性,在后续的“伪目标”章节中会有叙述。

\item[{prerequisites}] \leavevmode
生成该target所依赖的文件和/或target

\item[{command}] \leavevmode
该target要执行的命令(任意的shell命令)

\end{description}

这是一个文件的依赖关系,也就是说,target这一个或多个的目标文件依赖于prerequisites中的文件,
其生成规则定义在command中。说白一点就是说:

\fvset{hllines={, ,}}%
\begin{sphinxVerbatim}[commandchars=\\\{\}]
prerequisites中如果有一个以上的文件比target文件要新的话,command所定义的命令就会被执行。
\end{sphinxVerbatim}

这就是 \sphinxcode{\sphinxupquote{makefile}} 的规则,也就是 \sphinxcode{\sphinxupquote{makefile}} 中最核心的内容。

\sphinxcode{\sphinxupquote{echo "Hello World!";}}

行内公式使用 \sphinxcode{\sphinxupquote{math}} 这个 \sphinxcode{\sphinxupquote{role}}: \(a^2 + b^2 = c^2\).
\begin{equation*}
\begin{split}(a + b)^2  &=  (a + b)(a + b) \\
           &=  a^2 + 2ab + b^2\end{split}
\end{equation*}
\sphinxcode{\sphinxupquote{latex}} math测试:
\begin{equation*}
\begin{split}X_k =  \sum_{n=0}^{N-1} x_n e^{-{i 2\pi k \frac{n}{N}}} \qquad k = 0,\dots,N-1.\end{split}
\end{equation*}
将高亮语言设置为 \sphinxcode{\sphinxupquote{C}}

测试 \sphinxcode{\sphinxupquote{C}}

\fvset{hllines={, ,}}%
\begin{sphinxVerbatim}[commandchars=\\\{\},numbers=left,firstnumber=1,stepnumber=1]
\PYG{k+kt}{int} \PYG{n}{a} \PYG{o}{=} \PYG{l+m+mi}{0}\PYG{p}{;}
\PYG{k+kt}{char} \PYG{n}{c} \PYG{o}{=} \PYG{l+s+sc}{\PYGZsq{}}\PYG{l+s+sc}{c}\PYG{l+s+sc}{\PYGZsq{}}\PYG{p}{;}
\PYG{n}{printf}\PYG{p}{(}\PYG{l+s}{\PYGZdq{}}\PYG{l+s}{\PYGZpc{}c}\PYG{l+s+se}{\PYGZbs{}n}\PYG{l+s}{\PYGZdq{}}\PYG{p}{,} \PYG{n}{c}\PYG{p}{)}\PYG{p}{;}
\end{sphinxVerbatim}

这里是 \sphinxcode{\sphinxupquote{C++}} :

\fvset{hllines={, 3, 4, 5,}}%
\begin{sphinxVerbatim}[commandchars=\\\{\},numbers=left,firstnumber=1,stepnumber=1]
\PYG{k+kt}{int} \PYG{n+nf}{main}\PYG{p}{(}\PYG{p}{)}
\PYG{p}{\PYGZob{}}
  \PYG{k+kt}{int} \PYG{n}{i}\PYG{p}{;}
  \PYG{k+kt}{int} \PYG{n}{j}\PYG{p}{;}
  \PYG{n}{cin} \PYG{o}{\PYGZgt{}}\PYG{o}{\PYGZgt{}} \PYG{n}{i} \PYG{o}{\PYGZgt{}}\PYG{o}{\PYGZgt{}} \PYG{n}{j}\PYG{p}{;}
  \PYG{n}{cout} \PYG{o}{\PYGZlt{}}\PYG{o}{\PYGZlt{}} \PYG{n}{i} \PYG{o}{\PYGZlt{}}\PYG{o}{\PYGZlt{}} \PYG{n}{j} \PYG{o}{\PYGZlt{}}\PYG{o}{\PYGZlt{}} \PYG{n}{endl}\PYG{p}{;}
  \PYG{k}{return} \PYG{l+m+mi}{1}\PYG{p}{;}
\PYG{p}{\PYGZcb{}}
\PYG{c+c1}{// 主函数注释}
\end{sphinxVerbatim}

斜体 \sphinxtitleref{text}

将高亮语言设置为 \sphinxcode{\sphinxupquote{python}}

测试 \sphinxcode{\sphinxupquote{python}}

\fvset{hllines={, ,}}%
\begin{sphinxVerbatim}[commandchars=\\\{\},numbers=left,firstnumber=1,stepnumber=1]
\PYG{k+kn}{import} \PYG{n+nn}{torch}
\PYG{k+kn}{import} \PYG{n+nn}{numpy} \PYG{k+kn}{as} \PYG{n+nn}{np}
\PYG{k}{print} \PYG{l+s+s2}{\PYGZdq{}}\PYG{l+s+s2}{hello world}\PYG{l+s+s2}{\PYGZdq{}}
\end{sphinxVerbatim}

这里是 \sphinxcode{\sphinxupquote{python}} (code):

\fvset{hllines={, ,}}%
\begin{sphinxVerbatim}[commandchars=\\\{\},numbers=left,firstnumber=1,stepnumber=1]
\PYG{k}{def} \PYG{n+nf}{foo}\PYG{p}{(}\PYG{p}{)}\PYG{p}{:}
    \PYG{k}{print} \PYG{l+s+s2}{\PYGZdq{}}\PYG{l+s+s2}{Love Python, Love FreeDome}\PYG{l+s+s2}{\PYGZdq{}}
    \PYG{k}{print} \PYG{l+s+s2}{\PYGZdq{}}\PYG{l+s+s2}{E文标点,.0123456789,中文标点,. }\PYG{l+s+s2}{\PYGZdq{}}
\end{sphinxVerbatim}

如果数据库有问题, 执行下面的 \sphinxcode{\sphinxupquote{SQL}}:

\fvset{hllines={, ,}}%
\begin{sphinxVerbatim}[commandchars=\\\{\}]
\PYG{c+c1}{\PYGZhy{}\PYGZhy{} Dumping data for table {}`item\PYGZus{}table{}`}
\PYG{k}{INSERT} \PYG{k}{INTO} \PYG{n}{item\PYGZus{}table} \PYG{k}{VALUES} \PYG{p}{(}
\PYG{l+m+mi}{0000000001}\PYG{p}{,} \PYG{l+m+mi}{0}\PYG{p}{,} \PYG{l+s+s1}{\PYGZsq{}Manual\PYGZsq{}}\PYG{p}{,} \PYG{l+s+s1}{\PYGZsq{}\PYGZsq{}}\PYG{p}{,} \PYG{l+s+s1}{\PYGZsq{}0.18.0\PYGZsq{}}\PYG{p}{,}
\PYG{l+s+s1}{\PYGZsq{}This is the manual for Mantis version 0.18.0.\PYGZbs{}r\PYGZbs{}n\PYGZbs{}r\PYGZbs{}nThe Mantis manual is modeled after the [url=http://www.php.net/manual/en/]PHP Manual[/url]. It is authored via the \PYGZbs{}\PYGZbs{}\PYGZdq{}manual\PYGZbs{}\PYGZbs{}\PYGZdq{} module in Mantis CVS.  You can always view/download the latest version of this manual from [url=http://mantisbt.sourceforge.net/manual/]here[/url].\PYGZsq{}}\PYG{p}{,}
  \PYG{l+s+s1}{\PYGZsq{}\PYGZsq{}}\PYG{p}{,} \PYG{l+m+mi}{1}\PYG{p}{,} \PYG{l+m+mi}{1}\PYG{p}{,} \PYG{l+m+mi}{20030811192655}\PYG{p}{)}\PYG{p}{;}
\end{sphinxVerbatim}

下面是 \sphinxcode{\sphinxupquote{python}}:

\fvset{hllines={, 2, 3,}}%
\begin{sphinxVerbatim}[commandchars=\\\{\},numbers=left,firstnumber=1,stepnumber=1]
\PYG{c+c1}{\PYGZsh{} 测试注释}
\PYG{k}{def} \PYG{n+nf}{foo}\PYG{p}{(}\PYG{p}{)}\PYG{p}{:}
    \PYG{k}{print} \PYG{l+s+s2}{\PYGZdq{}}\PYG{l+s+s2}{Love Python, Love FreeDome}\PYG{l+s+s2}{\PYGZdq{}}
    \PYG{k}{print} \PYG{l+s+s2}{\PYGZdq{}}\PYG{l+s+s2}{E文标点,.0123456789,中文标点,. }\PYG{l+s+s2}{\PYGZdq{}}
\end{sphinxVerbatim}

下面是 \sphinxcode{\sphinxupquote{javescipt}} 的 \sphinxcode{\sphinxupquote{rst}} 源码:

\fvset{hllines={, ,}}%
\begin{sphinxVerbatim}[commandchars=\\\{\},numbers=left,firstnumber=1,stepnumber=1]
\PYG{o}{.}\PYG{o}{.} \PYG{n}{code}\PYG{p}{:}\PYG{p}{:} \PYG{n}{javascript}
    \PYG{p}{:}\PYG{n}{linenos}\PYG{p}{:}

    \PYG{n}{function} \PYG{n}{whatever}\PYG{p}{(}\PYG{p}{)}
    \PYG{p}{\PYGZob{}}
        \PYG{k}{return} \PYG{l+s+s2}{\PYGZdq{}}\PYG{l+s+s2}{such color}\PYG{l+s+s2}{\PYGZdq{}}
    \PYG{p}{\PYGZcb{}}
\end{sphinxVerbatim}

下面是 \sphinxcode{\sphinxupquote{python}} (code-block):

\fvset{hllines={, ,}}%
\begin{sphinxVerbatim}[commandchars=\\\{\},numbers=left,firstnumber=1,stepnumber=1]
\PYG{k}{def} \PYG{n+nf}{foo}\PYG{p}{(}\PYG{p}{)}\PYG{p}{:}
    \PYG{k}{print} \PYG{l+s+s2}{\PYGZdq{}}\PYG{l+s+s2}{Love Python, Love FreeDome}\PYG{l+s+s2}{\PYGZdq{}}
    \PYG{k}{print} \PYG{l+s+s2}{\PYGZdq{}}\PYG{l+s+s2}{E文标点,.0123456789,中文标点,. }\PYG{l+s+s2}{\PYGZdq{}}
\end{sphinxVerbatim}

下面是 \sphinxcode{\sphinxupquote{bash}} :

\fvset{hllines={, ,}}%
\begin{sphinxVerbatim}[commandchars=\\\{\},numbers=left,firstnumber=1,stepnumber=1]
\PYG{n+nb}{cd} home
\PYG{n+nb}{echo} \PYG{n+nv}{\PYGZdl{}PATH}
\PYG{n+nb}{source} \PYGZti{}/.bashrc
ls \PYGZhy{}l
mkdir filefolder
\PYG{n+nb}{cd} ..
\end{sphinxVerbatim}

结束



\renewcommand{\indexname}{Index}
\printindex
\end{document}